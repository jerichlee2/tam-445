\begin{problem}
    \noindent
    \normalfont

    \begin{enumerate}
        \item 
        \begin{enumerate}
            \item \begin{align}
                \psi=c_{1}(\text{tr(B)}-3), \text{B}=\text{FF}^{\text{T}} \\
                \end{align}
                
                
                Suppose $\text{T}$ is of the form:
                
                
                \begin{align}
                \text{T} = -p\mathbf{\text{I}}+ \square
                \end{align}
                
                
                The First Piola-Kirchhoff stress tensor, where $\text{J}=\det \text{F}$ is:
                
                
                \begin{align}
                \text{P}=\text{J}\text{T}\text{F}^{-\text{T}}
                \end{align}
                
                
                Another form of the Piola-Kirchhoff stress tensor, where $\hat{\text{W}} = \rho_{0}\psi$ is computed by the following:
                
                
                \begin{align}
                \text{P}^{e}=\frac{ \partial \hat{\text{W}} (\text{T},\text{F})}{ \partial \text{F} } 
                \end{align}
                
                Converting Eq 1 to indicial notation:
                
                \begin{align}
                \psi=c_{1}(\text{tr}(\text{F}_{iJ}F_{Ji})-3) \\
                =c_{1}(\text{F}\cdot \text{F}-3) \\
                c_{1}((\text{FF}^{\text{T}})_{ii}-3) \\
                =c_{1}(\text{F}_{iJ}\text{F}_{iJ}-3) \\
                \end{align}
                
                
                Applying Eq 4 to the above:
                
                
                \begin{align}
                \text{P}^{e}=\frac{ \partial \hat{\text{W}} (\text{T},\text{F})}{ \partial \text{F} }  \\
                =c_{1}\left( \text{F}_{iJ}\frac{ \partial \text{F}_{iJ} }{ \partial \text{F}_{kL} }  +\text{F}_{iJ}\frac{ \partial \text{F}_{iJ} }{ \partial \text{F}_{kL} }\right) \\
                =c_{1}\left(\text{F}_{iJ}\delta_{ik}\delta_{JL}+\text{F}_{iJ}\delta_{ik}\delta_{JL}\right) \\
                c_{1}\rho_{0}2\text{F}_{kL}
                \end{align}
                
                
                Substituting the above into the Cauchy Stress tensor:
                
                
                \begin{align}
                \text{T}=\frac{1}{\text{J}}\text{P}\text{F}^{\text{T}} \\
                = c_{1}\rho_{0}2\text{F}\text{F}^{\text{T}} \\
                =\underbrace{ c_{1}\rho_{0}2 }_{ \mu }\text{B} \\
                =\mu \text{B}
                \end{align}
                
                
                Therefore, we have shown that $\text{T}$ is of the form: 
                
                
                \begin{align}
                \text{T} = -p\mathbf{\text{I}}+ \mu \text{B}
                \end{align}
                \item The Cauchy stress due to the imposed simple shear is the following:
            
            \begin{align}
                \text{T} = -p\mathbf{\text{I}}+ \mu \text{B} \\
                =-p\underbrace{ \begin{bmatrix}
                1 & 0 & 0 \\
                0 & 1 & 0 \\
                0 & 0 & 1
                \end{bmatrix} }_{ \text{I} }+\mu \underbrace{ \begin{bmatrix}
                1 & \gamma(t) & 0 \\
                0 & 1 & 0 \\
                0 & 0 & 1
                \end{bmatrix}\begin{bmatrix}
                1 & 0 & 0 \\
                \gamma (t) & 1 & 0 \\
                0 & 0 & 1
                \end{bmatrix} }_{ \text{B} } \\
                =\begin{bmatrix}
                \mu^2(\gamma^2(t)+1)-p & \mu^2\gamma(t) & 0 \\
                \mu^2\gamma(t) & \mu^2-p & 0 \\
                0 & 0 & \mu^2-p
                \end{bmatrix}
                \end{align}
                
            \end{enumerate}
            \item The Cauchy stress tensor due to the imposed simple shear motion is given as the following:
            
            Computing the deformation map:


\begin{align}
\mathbf{x}=\begin{bmatrix}
\mathbf{X}_{1}+\mathbf{X}_{2}\gamma(t) \\
\mathbf{X}_{2} \\
\mathbf{X}_{3}
\end{bmatrix}
\end{align}


Computing $\mathbf{v}$:


\begin{align}
\mathbf{v}=\begin{bmatrix}
\mathbf{X}_{2}\dot{\gamma} \\
0 \\
0
\end{bmatrix}
\end{align}


Converting the above into the spatial description:


\begin{align}
\mathbf{v}_{s}=\begin{bmatrix}
\mathbf{x}_{2}\dot{\gamma} \\
0 \\
0
\end{bmatrix}
\end{align}

Computing the gradient of $\mathbf{v}_{s}$:


\begin{align}
\text{grad}\mathbf{v}_{s} = \begin{bmatrix}
0 & \dot{\gamma} & 0 \\
0 & 0 & 0 \\
0 & 0 & 0
\end{bmatrix}
\end{align}


The stretch rate tensor is given by:


\begin{align}
\text{D}_{s}=\frac{1}{2} \text{grad}\mathbf{v}_{s}+\text{grad}\mathbf{v}_{s}^{\text{T}} \\
=\frac{1}{2}\left(\begin{bmatrix}
0 & \dot{\gamma} & 0 \\
0 & 0 & 0 \\
0 & 0 & 0
\end{bmatrix}+\begin{bmatrix}
0 & 0 & 0 \\
\dot{\gamma} & 0 & 0 \\
0 & 0 & 0
\end{bmatrix}\right)
\end{align}


Substituting the above into the definition of the Cauchy stress for a Newtonian fluid:

\begin{align}
\text{T}=-p\text{I}+2\mu \text{D}_{s} \\
=-p\begin{bmatrix}
1 & 0 & 0 \\
0 & 1 & 0 \\
0 & 0 & 1
\end{bmatrix}+2\mu\left( \frac{1}{2} \text{grad}\mathbf{v}_{s}+\text{grad}\mathbf{v}_{s}^{\text{T}} \right) \\
=-p\begin{bmatrix}
1 & 0 & 0 \\
0 & 1 & 0 \\
0 & 0 & 1
\end{bmatrix}+2\mu\frac{1}{2}\left(\begin{bmatrix}
0 & \dot{\gamma} & 0 \\
0 & 0 & 0 \\
0 & 0 & 0
\end{bmatrix}+\begin{bmatrix}
0 & 0 & 0 \\
\dot{\gamma} & 0 & 0 \\
0 & 0 & 0
\end{bmatrix}\right) \\
= \begin{bmatrix}
-p & \mu \dot{\gamma} & 0 \\
\mu \dot{\gamma} & -p &0 \\
0 & 0 & -p
\end{bmatrix}
\end{align}



        \end{enumerate}
        
   


 
\end{problem}
