\begin{problem}
\normalfont
Compute the polar decompositions $
    \mathbf{T}=\mathbf{R}\mathbf{U}=\mathbf{V}\mathbf{R}
    $
    where $\mathbf{U},\mathbf{V} \in \text{Psym}$ and $\mathbf{R}\in \text{Orth}$, or a tensor $\mathbf{T}$ whose components are given by
    
    $$
    \begin{bmatrix}
    \sqrt{ 3 } & 1 & 0 \\
    0 & 2 & 0 \\
    0 & 0 & 1
    \end{bmatrix}
    $$
    
    Write down the eigenvectors and eigenvalues of $\mathbf{U}$ and $\mathbf{V}$, and describe in words the geometric interpretation of the above decompositions.
\end{problem}

$T^TT$ is found by multiplying the tensor $T$ by its transpose:


\begin{align*}
T^TT&=\begin{bmatrix}
\sqrt{ 3 } & 0 & 0 \\
1 & 2 & 0 \\
0 & 0 & 1
\end{bmatrix}\begin{bmatrix}
\sqrt{ 3 } & 1 & 0 \\
0 & 2 & 0 \\
0 & 0 & 1
\end{bmatrix}\\[10pt]
&=\begin{bmatrix}
3 & \sqrt{ 3 } & 0 \\
\sqrt{ 3 } & 5 & 0 \\
0 & 0 & 1
\end{bmatrix}
\end{align*}

To determine the eigenvalues and the eigenvalues pairs of symmetric tensor $T^TT$, the $\lambda$ values of $\text{det}(T-\lambda I)=0$ are found.


\begin{align*}
\text{det}(T-\lambda I)&=0\\[10pt]
&=\text{det}\begin{bmatrix}
3-\lambda & \sqrt{ 3 } & 0 \\
\sqrt{ 3 } & 5-\lambda & 0 \\
0 & 0 & 1-\lambda
\end{bmatrix}\\[10pt]
&=(3-\lambda)(5-\lambda)(1-\lambda)-\sqrt{ 3 }(\sqrt{ 3 }(1-\lambda))\\[10pt]
&=-\lambda^3+9\lambda^2-20\lambda+12=0
\end{align*}


Solving the polynomial equation, the following eigenvalues are determined:

$$
\lambda_{1}=1,\ \lambda_{2}=2,\ \lambda_{3}=6
$$

The procedure that follows involves determining the eigenvectors for each eigenvalue. 

Substituting $\lambda_{1}$ into $T-\lambda I$:


\begin{align*}
\begin{bmatrix}
2 & \sqrt{ 3 } & 0 \\
\sqrt{ 3 } & 4 & 0 \\
0 & 0 & 0
\end{bmatrix}\begin{bmatrix}
v_{1,1} \\
v_{2,1} \\
v_{3,1}
\end{bmatrix}&=\begin{bmatrix}
0 \\
0 \\
0
\end{bmatrix}
\end{align*}


The nullspace of $T-\lambda_{1}I$ is the space of which the eigenvector that corresponds to eigenvalue $\lambda_{1}$ lives in:

\begin{align*}
2v_{1,1}+\sqrt{ 3 }v_{1,2}&=0\\[10pt]
\sqrt{ 3 }v_{1,1}+4v_{1,2}&=0
\end{align*}


Setting our free variable equal to $v_{3}$, we get the following solution for eigenvalue $\lambda_{1}=1$:

$$
v_{1}=v_{1,3}\begin{bmatrix}
0 \\
0 \\
1
\end{bmatrix}
$$
This procedure is repeated for the remaining two eigenvalues, $\lambda_2$, and $\lambda_{3}$.

For $\lambda_{2}=2$:

$$
\begin{bmatrix}
1 & \sqrt{ 3 } & 0 \\
\sqrt{ 3 } & 3 & 0 \\
0 & 0 & -1
\end{bmatrix}\begin{bmatrix}
v_{2,1} \\
v_{2,2} \\
v_{2,3}
\end{bmatrix}=\begin{bmatrix}
0 \\
0 \\
0
\end{bmatrix}
$$

Computing the nullspace of the system above and setting the free variable equal to $v_{2,2}$:

\begin{align*}
v_{2,1}+\sqrt{ 3 }v_{2,2}&=0\\[10pt]
\sqrt{ 3 }v_{2,1}+3v_{2,2}&=0\\[10pt]
-v_{2,3}&=0\\[10pt]
\bar{v}_{2}&=\begin{bmatrix}
-\sqrt{ 3}v_{2,2} \\
v_{2,2} \\
0
\end{bmatrix}\\[10pt]
&=
v_{2,2}\begin{bmatrix}
-\sqrt{ 3 } \\
1 \\
0
\end{bmatrix}
\end{align*}




Substituting $\lambda_{3}$ into $\text{det}(T-\lambda I)$:

\begin{align*}
\begin{bmatrix}
-3 & \sqrt{ 3 } & 0 \\
\sqrt{ 3 } & -1 & 0 \\
0 & 0 & -5
\end{bmatrix}\begin{bmatrix}
v_{3,1} \\
v_{3,2} \\
v_{3,3}
\end{bmatrix}&=
\begin{bmatrix}
0 \\
0 \\
0
\end{bmatrix}
\end{align*}

Computing the nullspace of the above system and setting the free variable equal to $v_{3,2}$:

\begin{align*}
-3v_{3,1}+\sqrt{ 3 }v_{3,2}&=0\\[10pt]
\sqrt{ 3 }v_{3,1}-v_{3,2}&=0\\[10pt]
-5v_{3,3}&=0
\end{align*}

Solving for the nullspace:

\begin{align*}
\bar{v}_{3}&=\begin{bmatrix}
\frac{\sqrt{ 3 }}{3} \\
1 \\
0
\end{bmatrix}
\end{align*}


The eigenvalue and eigenvector pairs are shown below:

$$
\left( [1,\begin{bmatrix}
0 \\
0 \\
1
\end{bmatrix}], [2,\begin{bmatrix}
-\sqrt{ 3 } \\
1 \\
0
\end{bmatrix}],
\left[ 6,\begin{bmatrix}
\frac{\sqrt{ 3 }}{3} \\
0 \\
1
\end{bmatrix} \right] \right)
$$

Our next goal is to determine the polar decomposition $T=RU=VR$, where $U,V\in \text{Psym}$ and $R\in \text{Orth}$. 

$U$ is determined by the definition of the square root theorem, where $\underline{u}_{i}$ are normalized eigenvectors and $\lambda_{i}$ are the eigenvalues that were found in earlier steps:


\begin{align*}
U&=\sum_{i=1}^3 \sqrt{ \lambda_{i} }\underline{u}_{i}\otimes u_{i}\\[10pt]
&=\sqrt{ 1 }\begin{bmatrix}
0 \\
0 \\
1 \\
\end{bmatrix}\otimes 
\begin{bmatrix}
0 \\
0 \\
1
\end{bmatrix}+\sqrt{ 2 }\begin{bmatrix}
-\sqrt{ 3 } \\
1 \\
0
\end{bmatrix}\otimes \begin{bmatrix}
-\sqrt{ 3 } \\
1 \\
0
\end{bmatrix}+\sqrt{ 6 }\begin{bmatrix}
\frac{\sqrt{ 3 }}{3} \\
1 \\
0
\end{bmatrix}\otimes 
\begin{bmatrix}
\frac{\sqrt{ 3 }}{3} \\
1 \\
0
\end{bmatrix}\\[10pt]
&=\begin{bmatrix}
0 & 0 & 0 \\
0 & 0 & 0  \\
0 & 0 & 1
\end{bmatrix}+\sqrt{ 2 }\begin{bmatrix}
3 & -\sqrt{ 3 } & 0 \\
-\sqrt{ 3 } & 1 & 0 \\
0 & 0 & 0
\end{bmatrix}+\sqrt{ 6 }\begin{bmatrix}
\frac{1}{3} & \frac{\sqrt{ 3 }}{3} & 0 \\
\frac{\sqrt{ 3 }}{3} & 1 & 0 \\
0 & 0 & 0
\end{bmatrix}\\[10pt]
U&=\begin{bmatrix}
\frac{\sqrt{ 6 }}{4}+\frac{3\sqrt{ 2 }}{4} & -\frac{\sqrt{ 6 }}{4}+\frac{3\sqrt{ 2 }}{4} & 0 \\
-\frac{\sqrt{ 6 }}{4}+\frac{3\sqrt{ 2 }}{4} & \frac{\sqrt{ 2 }}{4}+\frac{3\sqrt{ 6 }}{4} & 0 \\
0 & 0 & 1
\end{bmatrix}
\end{align*}

Finding $R$ is straightforward using:

$$
R=TU^{-1}
$$

The inverse of $U$ can be found using the following:

\begin{align*}
U^{-1}&=\frac{1}{\text{det}(U)}\text{adj}(U)\\[10pt]
&=\begin{bmatrix}
\frac{\sqrt{ 6 }}{24}+\frac{3\sqrt{ 2 }}{8} & -\frac{\sqrt{ 6 }}{8}+\frac{\sqrt{ 2 }}{8} & 0 \\
-\frac{\sqrt{ 6 }}{8}+\frac{\sqrt{ 2 }}{8} & \frac{\sqrt{ 2 }}{8}+\frac{\sqrt{ 6 }}{8} & 0 \\
0 & 0 & 1
\end{bmatrix}
\end{align*}

Computing $R$:

\begin{align*}
R&=TU^{-1}\\[10pt]
&= \begin{bmatrix}
\sqrt{ 3 } & 1 & 0 \\
0 & 2 & 0 \\
0 & 0 & 1
\end{bmatrix}\begin{bmatrix}
\frac{\sqrt{ 6 }}{24}+\frac{3\sqrt{ 2 }}{8} & -\frac{\sqrt{ 6 }}{8}+\frac{\sqrt{ 2 }}{8} & 0 \\
-\frac{\sqrt{ 6 }}{8}+\frac{\sqrt{ 2 }}{8} & \frac{\sqrt{ 2 }}{8}+\frac{\sqrt{ 6 }}{8} & 0 \\
0 & 0 & 1
\end{bmatrix}\\[10pt]
R&=\begin{bmatrix}
\frac{\sqrt{ 2 }}{4}+\frac{\sqrt{ 6 }}{4} & -\frac{\sqrt{ 2 }}{4}+\frac{\sqrt{ 6 }}{4} & 0 \\
-\frac{\sqrt{ 6 }}{4}+\frac{\sqrt{ 2 }}{4} & \frac{\sqrt{ 2 }}{4}+\frac{\sqrt{ 6 }}{4} & 0 \\
0 & 0 & 1
\end{bmatrix}
\end{align*}

Using the definition $V=RUR^T$:

\begin{align*}
V&=RUR^T\\[10pt]
&=\begin{bmatrix}
\frac{\sqrt{ 2 }}{4}+\frac{\sqrt{ 6 }}{4} & -\frac{\sqrt{ 2 }}{4}+\frac{\sqrt{ 6 }}{4} & 0 \\
-\frac{\sqrt{ 6 }}{4}+\frac{\sqrt{ 2 }}{4} & \frac{\sqrt{ 2 }}{4}+\frac{\sqrt{ 6 }}{4} & 0 \\
0 & 0 & 1
\end{bmatrix}\begin{bmatrix}
\frac{\sqrt{ 6 }}{4}+\frac{3\sqrt{ 2 }}{4} & -\frac{\sqrt{ 6 }}{4}+\frac{3\sqrt{ 2 }}{4} & 0 \\
-\frac{\sqrt{ 6 }}{4}+\frac{3\sqrt{ 2 }}{4} & \frac{\sqrt{ 2 }}{4}+\frac{3\sqrt{ 6 }}{4} & 0 \\
0 & 0 & 1
\end{bmatrix}\begin{bmatrix}
\frac{\sqrt{ 2 }}{4}+\frac{\sqrt{ 6 }}{4} & -\frac{\sqrt{ 6 }}{4}+\frac{\sqrt{ 2 }}{4} & 0 \\
-\frac{\sqrt{ 2 }}{4}+\frac{\sqrt{ 6 }}{4} & \frac{\sqrt{ 2 }}{4}+\frac{\sqrt{ 6 }}{4} & 0 \\
0 & 0 & 1
\end{bmatrix}\\[10pt]
V&=\begin{bmatrix}
\frac{\sqrt{ 2 }}{2}+\frac{\sqrt{ 6 }}{2} & -\frac{\sqrt{ 2 }}{2}+\frac{\sqrt{ 6 }}{2} & 0 \\
-\frac{\sqrt{ 2 }}{2}+\frac{\sqrt{ 6 }}{2} & \frac{\sqrt{ 2 }}{2}+\frac{\sqrt{ 6 }}{2} & 0 \\
0 & 0 & 1
\end{bmatrix}
\end{align*}


The tensor $T$ can be decomposed into the the polar decomposition $T=RU$ and $T=VR$ as the following:

Right Polar Decomposition:

\begin{align*}
T&=RU\\[10pt]
&=\begin{bmatrix}
\frac{\sqrt{ 2 }}{4}+\frac{\sqrt{ 6 }}{4} & -\frac{\sqrt{ 2 }}{4}+\frac{\sqrt{ 6 }}{4} & 0 \\
-\frac{\sqrt{ 6 }}{4}+\frac{\sqrt{ 2 }}{4} & \frac{\sqrt{ 2 }}{4}+\frac{\sqrt{ 6 }}{4} & 0 \\
0 & 0 & 1
\end{bmatrix}\begin{bmatrix}
\frac{\sqrt{ 6 }}{4}+\frac{3\sqrt{ 2 }}{4} & -\frac{\sqrt{ 6 }}{4}+\frac{3\sqrt{ 2 }}{4} & 0 \\
-\frac{\sqrt{ 6 }}{4}+\frac{3\sqrt{ 2 }}{4} & \frac{\sqrt{ 2 }}{4}+\frac{3\sqrt{ 6 }}{4} & 0 \\
0 & 0 & 1
\end{bmatrix}\\[10pt]
&=\begin{bmatrix}
\sqrt{ 3 } & 1 & 0 \\
0 & 2 & 0 \\
0 & 0 & 1
\end{bmatrix}
\end{align*}

Left Polar Decomposition:

\begin{align*}
T&=VR\\[10pt]
&=\begin{bmatrix}
\frac{\sqrt{ 2 }}{2}+\frac{\sqrt{ 6 }}{2} & -\frac{\sqrt{ 2 }}{2}+\frac{\sqrt{ 6 }}{2} & 0 \\
-\frac{\sqrt{ 2 }}{2}+\frac{\sqrt{ 6 }}{2} & \frac{\sqrt{ 2 }}{2}+\frac{\sqrt{ 6 }}{2} & 0 \\
0 & 0 & 1
\end{bmatrix}\begin{bmatrix}
\frac{\sqrt{ 2 }}{4}+\frac{\sqrt{ 6 }}{4} & -\frac{\sqrt{ 2 }}{4}+\frac{\sqrt{ 6 }}{4} & 0 \\
-\frac{\sqrt{ 6 }}{4}+\frac{\sqrt{ 2 }}{4} & \frac{\sqrt{ 2 }}{4}+\frac{\sqrt{ 6 }}{4} & 0 \\
0 & 0 & 1
\end{bmatrix}\\[10pt]
&=
\begin{bmatrix}
\sqrt{ 3 } & 1 & 0 \\
0 & 2 & 0 \\
0 & 0 & 1
\end{bmatrix}
\end{align*}

The geometric interpretation of the polar decomposition of tensor $T$ consists of a rotation and a dilation, where $R$ is a rotation matrix, taking a unit square and operating upon the square as a rigid body rotation without deformation. The tensors $U$ and $V$ perform stretch operations upon a unit square. 


