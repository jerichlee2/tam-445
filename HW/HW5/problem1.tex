\begin{problem}
    \noindent
    \normalfont
\end{problem}
    \begin{enumerate}
        \item $\nabla \mathbf{v}=\mathbf{I}$
        \begin{proof}


        
        We begin by using the definition of the gradient of a vector field:
        \begin{align}
            \nabla \mathbf{v}=\frac{ \partial x_{i} }{ \partial x_{j} } \underline{e}_{i}\otimes \underline{e}_{j} \\
            \mathbf{I} = \delta_{ij}\underline{e}_{i}\otimes \underline{e}_{j} \\
            \end{align}
        Applying the identity tensor to an arbitrary vector $\mathbf{u}$:
        \begin{align}
            \mathbf{I}\underline{u} = (\delta_{ij}\underline{e}_{i}\otimes \underline{e}_{j})\underline{u} \\
            (\mathbf{I}\underline{u})_{i}=\mathbf{I}_{ij}u_{j} \label{eq:5} \\
            (\delta_{ij}\underline{e}_{i}\otimes \underline{e}_{j})\underline{u}=\delta_{ij}(\underline{e}_{j}\cdot \underline{u})\underline{e}_{i} \\
            =\delta_{ij}u_{j}\underline{e}_{i} \label{eq:7}
        \end{align}

        Using the definiton of the identity tensor and equating Eqns (\ref{eq:5}) and (\ref{eq:7}):
        \begin{align}
            \mathbf{I}_{ij}u_{j}=\delta_{ij}u_{j} \label{eq:8}
            \end{align}

        The identity tensor is equal to the Kronecker delta, as desired.
        \end{proof}

        \item $\nabla v=\frac{\mathbf{x}}{x}$
        \begin{proof}
                Using the given definition of $x$ and $v$:
            \begin{align}
                v(\mathbf{x})=(x_{i}x_{i})^{1/2}\label{eq:9}
                \end{align}
            Substituting Eqn (\ref{eq:9}) into the definition of the gradient of a scalar field:
            \begin{align}
                \frac{ \partial (x_{j}x_{j})^{1/2} }{ \partial x_{i} } =\frac{1}{2}(x_{j}x_{j})^{-1/2}\left( \frac{ \partial x_{j} }{ \partial x_{i} } x_{j} +\frac{ \partial x_{j} }{ \partial x_{i} } x_{j}\right) \\
                =\frac{1}{2}(x_{j}x_{j})^{-1/2}\left( 2\frac{ \partial x_j }{ \partial x_{i} } x_{j} \right) \\
                =\frac{\delta_{ij}x_{j}}{(x_{j}x_{j})^{1/2}} \\
                =\frac{x_{i}}{(x_{j}x_{j})^{1/2}} \\
            \end{align}
        The RHS is equal to the RHS of the problem, as desired.
        \end{proof}
        \pagebreak
        \item $\text{div}(\mathbf{v})=3$
        \begin{proof}
            \begin{align}
                \text{div}(\mathbf{v})=\text{tr}(\nabla \mathbf{v}) \label{eq:15}\\
                =\text{tr}(\mathbf{I}) \label{eq:16} \\
                =3
                \end{align}
                Eqn (\ref{eq:16}) is implied by Eqn (\ref{eq:8}), and the trace of the identity is $3$, as desired.
        \end{proof}
        \item $\text{div}\left( \frac{\mathbf{v}}{v} \right)=\frac{2}{x}$
        \begin{proof}
            We begin by setting $\phi$ equal to the scalar field of the domain of the LHS of the given problem:
            \begin{align}
                \phi=\frac{1}{v} =\frac{1}{(v_{i}v_{i})^{1/2}} \label{eq:18}
                \end{align}
            Eqn (\ref{eq:18}) is substituted into the following identity:
            \begin{align}
                \text{div}(\phi \mathbf{v})=\phi \ \text{div}(\mathbf{v})+\mathbf{v}\cdot \nabla \phi\\
                =\frac{1}{v}\text{tr}(\nabla \mathbf{v})+\mathbf{v}\cdot \left( \nabla \frac{1}{v} \right) \label{eq:20}
            \end{align}
            We observe the following statement from the given information:
            \begin{align}
                \lvert \mathbf{x} \rvert =(\mathbf{x}\cdot \mathbf{x})^{1/2} \label{eq:21} \\
                =x \label{eq:22} \\
                =v \label{eq:23}
                \end{align}
            Simplifying the first term of Eqn (\ref{eq:20}) using Eqns (\ref{eq:21}) through (\ref{eq:23}):
                \begin{align}
                    \frac{1}{v}\text{tr}(\mathbf{I}) =\frac{3}{x} \label{eq:24} 
                    \end{align}
            The term inside of the parentheses in the second term of Eqn (\ref{eq:20}) is simplified as follows:
            \begin{align}
                \nabla \frac{1}{v} \\
                =\nabla (x_{i}x_{i})^{-1/2} \\
                = \frac{ \partial (x_{j}x_{j})^{-1/2} }{ \partial x_{i} }  \\
                =-\frac{1}{2}(x_{j}x_{j})^{-3/2}(\delta_{ij}x_{j}+\delta_{ij}x_{j}) \\
                =-\frac{2\delta_{ij}x_{j}}{2(x_{j}x_{j})^{3/2}} \\
                =-\frac{x_{i}}{(x_{j}x_{j})^{3/2}} \\
                =-\frac{\mathbf{x}}{x^3} \label{eq:31}
                \end{align}
                Substituting Eqn (\ref{eq:31}) into the second term of Eqn (\ref{eq:20}):
                \begin{align}
                    \mathbf{x}\cdot\left( -\frac{\mathbf{x}}{x^3} \right) \\
                    =\frac{1}{x^3}(\mathbf{x}\cdot \mathbf{x}) \\
                    =-\frac{1}{x^3}(x^2) \\
                    =-\frac{1}{x} \label{eq:36}
                    \end{align}
            
            Substituting Eqns (\ref{eq:36}) and (\ref{eq:24}) into Eqn (\ref{eq:20}):
            \begin{align}
                \frac{3}{x}+\mathbf{x}\cdot\left( -\frac{\mathbf{x}}{x^3} \right) \\
                =\frac{2}{x}
                \end{align}
        \end{proof}
        \item $\text{div}(\mathbf{v}\otimes \mathbf{v})=4x$
        \begin{proof}
            Using the following identity:
            \begin{align}
                \text{div}(\mathbf{v}\otimes \mathbf{v})=\mathbf{v}(\text{div}(\mathbf{v}))+(\nabla \mathbf{v})\mathbf{v}
                \end{align}
            Simplifying:
            \begin{align} \\
                \mathbf{v}(\text{div}(\mathbf{v}))+(\nabla \mathbf{v})\mathbf{v} =
                3\mathbf{v}+I\mathbf{v} \\
                =4\mathbf{v} \\
                =4\mathbf{x}
                \end{align}
        \end{proof}
        \pagebreak
        \item $\text{curl}(\mathbf{v})=\mathbf{0}$
        \begin{proof}
            We start with the definition of the curl of a vector in indicial notation:
            \begin{align}
                \text{curl}(\mathbf{v})=\epsilon_{kji}\frac{ \partial v_{i} }{ \partial x_{j} } e_{k} \label{eq:43}
                \end{align}
            Using Eqn (\ref{eq:21}), Eqn (\ref{eq:43}) is simplified as follows:
            \begin{align}
                =\epsilon_{kji}\frac{ \partial v_{i} }{ \partial x_{j} } e_{k} \\
                =\epsilon_{kji}\delta_{ij}e_{k} \\
                =\epsilon_{kii}e_{k} \\
                =\mathbf{0} \label{eq:47}
                \end{align}
        \end{proof}
        \item $\text{curl}\left( \frac{\mathbf{v}}{v} \right)=\mathbf{0}$
        \begin{proof}
            Setting $\phi=\frac{1}{v}$, using the identity of the curl of a scalar field multiplied by a vector field, and Eqn (\ref{eq:47}):
            \begin{align}
            \frac{1}{v}=\phi \\
            =\underbrace{ \cancel{ \phi \ \text{curl}(\mathbf{v}) } }_{ 0 }+\nabla \phi \times \mathbf{v}\label{eq:49}
            \end{align}
            Substituting Eqn (\ref{eq:31}) into Eqn (\ref{eq:49}):
            \begin{align}
            =-\left( \frac{1}{x^3} \right)\underbrace{ \cancel{ \mathbf{x}\times \mathbf{x} } }_{ 0 } \\
            =\mathbf{0}
            \end{align}

        \end{proof}
    \end{enumerate}
   \pagebreak 
