\begin{problem}
    \noindent
    \normalfont
\end{problem}
    \begin{enumerate}
        \item
        \begin{enumerate}
            
        \item $\text{div}(\text{curl}(\mathbf{v}))=0$
            \begin{proof}
                \begin{align}
                    \text{div}(\text{curl}(\mathbf{v}))=\text{div}\left( \epsilon_{kji}\frac{ \partial v_{i} }{ \partial x_{j} } \underline{e}_{k} \right) \\
                    \end{align}
                    
                    Setting $\mathbf{\gamma}(\mathbf{v})=\underline{e}_{k}$ and $\varepsilon(\mathbf{v})=\epsilon_{kji}\frac{ \partial v_{i} }{ \partial x_{j} }$ and using the identity of divergence of the curl of a vector:
        
                    \begin{align}
                    \underbrace{ \cancel{ \varepsilon \ \text{tr}(\nabla \underline{e}_{k}) } }_{ 0 }+\underbrace{ \cancel{ \underline{e}_{k}\cdot \frac{ \partial \varepsilon }{ \partial x_{i} } \underline{e}_{i} } }_{ 0 }    
                    \end{align}
            \end{proof}
            


        \item $\text{curl}(\nabla \phi)=\mathbf{0}$
        \begin{proof}
            \begin{align}
                \text{curl}(\nabla \phi)=\mathbf{0} \\
                \text{curl}\left( \frac{ \partial \phi }{ \partial x_{i} } \underline{e}_{i} \right) \\
                =\nabla \times\left( \frac{ \partial \phi }{ \partial x_{i} } \underline{e}_{i} \right) \\
                =\underbrace{ \cancel{ \nabla \frac{ \partial \phi }{ \partial x_{i} } \times \underline{e}_{i} } }_{ 0 }+\underbrace{ \cancel{ \frac{ \partial \phi }{ \partial x_{i} } \text{curl}(\underline{e}_{i}) } }_{ 0 }
                \end{align}
        \end{proof}

        \end{enumerate}
        \item
        \begin{enumerate}
            \item $\nabla \phi+\text{curl}(\mathbf{v})=\mathbf{0}$
            \begin{proof}
                We begin by calculating $\nabla \phi$:

                \begin{align}
                \phi=\frac{c_{j}x_{j}}{(x_{k}x_{k})^{3/2}} \\
                \nabla \phi=\frac{ \partial (c_{j}x_{j}(x_{k}x_{k})^{-3/2}) }{ \partial x_{i} } \underline{e}_{k} \\
                =c_{j}\frac{ \partial x_{j} }{ \partial x_{i} } (x_{k}x_{k})^{-3/2}\underline{e}_{k}+ -\frac{3}{2}c_{j}x_{j}(x_{k}x_{k})^{-5/2}\left( 2\frac{ \partial x_{k} }{ \partial x_{i} } x_{k} \right)\underline{e}_{k} \\
                =\frac{c_{i}}{(x_{k}x_{k})^{3/2}}\underline{e}_{i}-\frac{3c_{j}x_{j}x_{i}}{(x_{k}x_{k})^{5/2}}\underline{e}_{i} \label{eq:62}
                \end{align}
                Calculating $\mathbf{v}$ in indicial notation:

\begin{align}
\mathbf{v}(\mathbf{x})=\frac{\mathbf{c}\times \mathbf{v}}{x^3} \\
=\frac{\epsilon_{ijk}c_{j}x_{k}}{(x_{s}x_{s})^{3/2}}\underline{e}_{i}
\end{align}

Calculating $\text{curl}(\mathbf{v})$:


\begin{align}
\nabla \times \mathbf{v}=\epsilon_{lnm}\frac{ \partial (\epsilon_{ijk}c_{j}x_{k}(x_{s}x_{s})^{-3/2}) }{ \partial x_{n} } \delta_{\mathrm{im}}\underline{e}_{l} \label{eq:65} \\
\end{align}


Taking the partial derivative of Eqn (\ref{eq:65}) using the product rule for the first term:


\begin{align}
=\epsilon_{lnm}\epsilon_{ijk}c_{j}\delta_{kn}(x_{s}x_{s})^{-3/2}\delta_{im}\underline{e}_{l} \\
\epsilon_{lni}\epsilon_{jni}c_{j}(x_{s}x_{s})^{-3/2}\underline{e}_{l} \\
=2\delta_{lj}c_{j}(x_{s}x_{s})^{-3/2}\underline{e}_{l} \\
=\frac{2c_{l}}{(x_{s}x_{s})^{3/2}}
\end{align}

Taking the partial derivative of Eqn (\ref{eq:65}) using the product rule for the second term:

\begin{align}
 -\frac{3}{2} \epsilon_{lnm}\epsilon_{ijk}c_{j}x_{k}(x_{s}x_{s})^{-5/2}(2\delta_{sn}x_{s})\delta_{im}e_{l} \\
=-\frac{3}{2}\epsilon_{lni}\epsilon_{ijk}c_{j}x_{k}(x_{s}x_{s})^{-5/2}2x_{n}e_{l} \\
=-3\epsilon_{lni}\epsilon_{jki}c_{j}x_{k}x_{n}(x_{s}x_{s})^{-5/2}\underline{e}_{l} \\
(\delta_{lj}\delta_{nk}-\delta_{lk}\delta_{nj})(-3c_{j}x_{k}x_{n}(x_{s}x_{s})^{-5/2})\underline{e}_{l} \\
=\delta_{lj}\delta_{nk}(-3c_{j}x_{k}x_{n}(x_{s}x_{s})^{-5/2})\underline{e}_{l}-\delta_{lk}\delta_{nj}(-3c_{j}x_{k}x_{n}(x_{s}x_{s})^{-5/2})\underline{e}_{l} \\
=-3c_{l}x_{k}x_{k}(x_{s}x_{s})^{-5/2}e_{l}+3c_{n}x_{l}x_{n}(x_{s}x_{s})^{-5/2}\underline{e}_{l} \\
=-\frac{3c_{l}}{(x_{s}x_{s})^{3/2}}+\frac{3c_{n}x_{n}x_{l}}{(x_{s}x_{s})^{5/2}}+\frac{2c_{l}}{(x_{s}x_{s})^{3/2}} \\
=-\frac{c_{l}}{(x_{s}x_{s})^{3/2}}+\frac{3c_{n}x_{n}x_{l}}{(x_{s}x_{s})^{5/2}} \label{eq:78}
\end{align}
Adding Eqns (\ref{eq:62}) and (\ref{eq:78}) together:
\begin{align}
    \nabla \phi+\text{curl}(\mathbf{v})=\mathbf{0} \\
    =
    \frac{c_{i}}{(x_{k}x_{k})^{3/2}}\underline{e}_{i}-\frac{3c_{j}x_{j}x_{i}}{(x_{k}x_{k})^{5/2}}\underline{e}_{i}-\frac{c_{l}}{(x_{s}x_{s})^{3/2}}\underline{e}_{l}+\frac{3c_{n}x_{n}x_{l}}{(x_{s}x_{s})^{5/2}}\underline{e}_{l} \\
    =0
    \end{align}
            \end{proof}
\pagebreak
            \item $\Delta \phi=0$
            \begin{proof}
                \begin{align}
                    \Delta \phi=0 \\
                    \phi(\mathbf{x})=\frac{\mathbf{c}\cdot \mathbf{x}}{x^3} \\
                    \nabla \phi=\left( \frac{c_{i}}{(x_{k}x_{k})^{3/2}}-\frac{3c_{j}x_{j}x_{i}}{(x_{k}x_{k})^{5/2}} \right)\underline{e}_{i} \\
                    \nabla(\nabla \phi)=\nabla\left( \frac{c_{i}}{(x_{k}x_{k})^{3/2}}\right)-\nabla\left( \frac{3c_{j}x_{j}x_{i}}{(x_{k}x_{k})^{5/2}}  \right) \\
                    =\nabla(c_{i}(x_{k}x_{k})^{-3/2})-\nabla(3c_{j}x_{j}x_{i}(x_{k}x_{k})^{-5/2}) \\
                    \frac{ \partial c_{i}(x_{k}x_{k})^{-3/2} }{ \partial x_{j} } -\frac{ \partial (3c_{j}x_{j}x_{i}(x_{k}x_{k})^{-5/2}) }{ \partial x_{j} } 
                    \end{align}
Simplifying further:
\begin{align}
=-\frac{3}{2}c_{i}(x_{k}x_{k})^{-5/2}(2\delta_{kj}x_{k})-3c_{j}(3)x_{i}(x_{k}x_{k})^{-5/2} \\
=-3c_{i}(x_{k}x_{k})^{-5/2}x_{j}=\frac{9c_{j}x_{i}}{(x_{k}x_{k})^{5/2}}+\frac{3c_{j}x_{i}}{(x_{k}x_{k})^{5/2}} \\
-\frac{3c_{i}x_{j}}{(x_{k}x_{k})^{5/2}}=3c_{j}x_{j}x_{i}\left( -\frac{5}{2} \right)(x_{k}x_{k})^{-7/2}(2\delta_{kj}x_{k}) \\
=-3c_{i}x_{j} \\
(x_{k}x_{k})^{5/2}=-\frac{15c_{j}(x_{j}x_{j})x_{i}}{(x_{k}x_{k})^{7/2}} \\
=\frac{-3c_{i}x_{j}}{(x_{k}x_{k})^{5/2}}+\frac{3c_{j}x_{i}}{(x_{k}x_{k})^{5/2}}
\end{align}

The index $i=j$ when applying the trace function upon a tensor, so the following holds true:

\begin{align}
\text{tr}\left( \frac{-3c_{i}x_{j}}{(x_{k}x_{k})^{5/2}}+\frac{3c_{j}x_{i}}{(x_{k}x_{k})^{5/2}} \right) \\
=0
\end{align}
            \end{proof}
            \pagebreak
            \item $\text{div}(\mathbf{v})$
            
\begin{align}
\text{div}(\mathbf{v})=0 \\
\text{div}(\epsilon_{ijk}c_{j}x_{k}(x_{s}x_{s})^{-3/2})\underline{e}_{i} \\
=tr(\nabla \mathbf{v}) \\
=\frac{ \partial \epsilon_{ijk}c_{j}x_{k}(x_{s}x_{s}^{-3/2}) }{ \partial x_{j} } \underline{e}_{i}\otimes \underline{e}_{j} \\
=\epsilon_{ijk}c_{j}\delta_{kj}(x_{s}x_{s})^{-3/2}+\epsilon_{ijk}c_{j}x_{k}\left( -\frac{3}{2} \right)(x_{s}x_{s})^{5/2}(2\delta_{sj}x_{s}) \\
=\underbrace{ \cancel{ \epsilon_{ikk}c_{j}(x_{s}x_{s})^{-3/2} } }_{ 0 }-\frac{3\epsilon_{ijk}c_{j}x_{k}x_{j}}{(x_{s}x_{s})^{5/2}}\underline{e}_{i}\otimes \underline{e}_{j} \label{eq:101}
\end{align}

Taking the trace of Eqn (\ref{eq:101}):

\begin{align}
\text{tr}(\nabla \mathbf{v})=-\frac{3\epsilon_{ijk}c_{j}x_{k}x_{j}}{(x_{s}x_{s})^{5/2}}\delta_{ij} \\
=-\frac{3\epsilon_{iik}c_{j}x_{j}x_{k}}{(x_{s}x_{s})^{5/2}} \\
=0
\end{align}
        \end{enumerate}
        \pagebreak
        \item Plots, where $\mathbf{c} = (1, 1, 1)$:
\begin{figure}[htbp]
    \centering
    \incfig{plot1}
    \caption{Plot of $\mathbf{v}(\mathbf{x})$ in Problem 1}
    \label{fig:plot1}
\end{figure}
\begin{figure}[htbp]
    \centering
    \incfig{phiplot}
    \caption{Plot of $\mathbf{v}(\mathbf{x})$ in Problem 2}
    \label{fig:phiplot}
\end{figure}
\begin{figure}[htbp]
    \centering
    \incfig{vplot}
    \caption{Plot of $\phi(\mathbf{x})$ in Problem 2}
    \label{fig:vplot}
\end{figure}
         \end{enumerate}