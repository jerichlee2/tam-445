\begin{problem}
    \noindent
    \normalfont

    We can show that the deformation mapping Eqn (\ref{deform}) preserves planes by determining the change in surface area using the general point $\mathbf{p}=(a,b,c)$. Using Nanson's Formula (\ref{nanson}) for the general point $\mathbf{p}$:
    \begin{align}
    \hat{n'}=\mathbf{F}\hat{e}_{2}\times \mathbf{F}\hat{e}_{3}=\det \mathbf{F}F^{-T}\hat{n} \\
    =\left( \frac{b+r}{r} \right)\left( \begin{bmatrix}
    \left( -b+r \right)\cos\left( \frac{a}{r} \right)&-\frac{\left( -b+r \right)\sin\left( \frac{a}{r} \right)}{r} & 0 \\
    \sin\left( \frac{a}{r} \right) & \cos\left( \frac{a}{r} \right) & 0 \\
    0 & 0 & 1
    \end{bmatrix} \right)\begin{bmatrix}
    1 \\
    0 \\
    0
    \end{bmatrix} \\
    = \begin{bmatrix}
    \frac{(b+r)^2\cos\left( \frac{a}{r} \right)}{r^2} \\
    \frac{(b+r)\sin\left( \frac{a}{r} \right)}{r}
    0
    \end{bmatrix} \label{generalnanson}
    \end{align}
    
    Eqn (\ref{generalnanson}) is only dependent upon two parameters, namely $a$ and $b$. Therefore, points on a surface in the reference configuration will remain on the same plane on the deformed configuration.\end{problem}
