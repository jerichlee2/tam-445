\begin{problem}
    \noindent
    \normalfont

    The mid-plane is described as the point $\mathbf{p}=(l,0, l)$. Computing the Lagrangian strain tensor with $\mathbf{p}$ as the input:


\begin{align}
\mathbf{E}=\frac{\mathbf{F}^{\text{T}}\mathbf{F}-\mathbf{I}}{2} \\
= \frac{\left( \begin{bmatrix}
\cos\left( \frac{l}{r} \right) & \sin\left( \frac{l}{r} \right) & 0 \\
-\sin\left( \frac{l}{r} \right) & \cos\left( \frac{l}{r} \right) & 0 \\
0 & 0 & 1
\end{bmatrix}\begin{bmatrix}
\cos\left( \frac{l}{r} \right) & -\sin\left( \frac{l}{r} \right) & 0 \\
\sin\left( \frac{l}{r} \right) & \cos\left( \frac{l}{r} \right) & 0 \\
0 & 0 & 1
\end{bmatrix}-\begin{bmatrix}
1 & 0 & 0 \\
0 & 1 & 0 \\
0 & 0 & 1
\end{bmatrix} \right)}{2} \\
=\begin{bmatrix}
0 & 0 & 0 \\
0 & 0 & 0 \\
0 & 0 & 0
\end{bmatrix} \label{lag_strain}
\end{align}
\begin{figure}[htbp]
    \centering
    \incfig{midplane}
    \caption{The mid-plane represented by point $\mathbf{p}$}
    \label{fig:midplane}
\end{figure}
Eqn (\ref{lag_strain}) tells us that there is no strain on the mid-plane after the deformation of the reference configuration. This makes sense because the mid-plane can be treated as the neutral axis passing through the centroid of the reference configuration. 
\end{problem}