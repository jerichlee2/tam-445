\begin{problem}
    \noindent
    \normalfont

    We will compute the infinitesimal strain tensor on the mid-plane $\mathbf{p}=(l,0,l)$. First, we will compute the displacement field $\mathbf{u}(\mathbf{p})=\mathbf{x}(\mathbf{p})-\mathbf{p}$:

\begin{align}
\mathbf{u}(\mathbf{p})=\mathbf{x(p)}-\mathbf{p} \\
= \begin{bmatrix}
r\sin\left( \frac{l}{r} \right) \\
r\left( \cos\left( \frac{l}{r} \right)-1 \right) \\
l
\end{bmatrix}-\begin{bmatrix}
l \\
0 \\
l
\end{bmatrix} \\
=\begin{bmatrix}
-l+r\sin\left( \frac{l}{r} \right) \\
-r\left( 1-\cos\left( \frac{l}{r} \right) \right) \\
0
\end{bmatrix}
\end{align}

Computing the infinitesimal strain tensor:


\begin{align}
\mathbf{\epsilon}=\frac{\nabla \mathbf{u}+\nabla \mathbf{u}^{\text{T}}}{2} \\
=\frac{\left( \begin{bmatrix}
\cos\left( \frac{l}{r} \right)-1 & \sin\left( \frac{l}{r} \right) & 0 \\
-\sin\left( \frac{l}{r} \right) & \cos\left( \frac{l}{r} \right)-1 & 0 \\
0 & 0 & 0
\end{bmatrix}+\begin{bmatrix}
\cos\left( \frac{l}{r} \right)-1 & -\sin\left( \frac{l}{r} \right) & 0 \\
\sin\left( \frac{l}{r} \right) & \cos\left( \frac{l}{r} \right)-1 & 0 \\
0 & 0 & 0
\end{bmatrix} \right)}{2} \\
=\begin{bmatrix}
\cos\left( \frac{l}{r} \right)-1 & 0 & 0 \\
0 & \cos\left( \frac{l}{r} \right)-1 & 0 \\
0 & 0 & 0
\end{bmatrix}\label{inf}
\end{align}

The infinitesimal strain tensor (\ref{inf}) is non-zero on the mid-plane, while the Lagrangian strain tensor (\ref{lag_strain}) is zero on the mid-plane. The infinitesimal strain tensor  shows that when $r\gg l$, then $\mathbf{\epsilon}\to 0$, meaning that when the radius of curvature is much larger than the length of the surface along axis $1$, the infinitesimal strain approaches zero. Therefore, the infinitesimal strain tensor is a good approximation of the Lagrangian strain tensor when $r\gg l$.
\end{problem}